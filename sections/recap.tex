% Recap
% Roberto Masocco <roberto.masocco@uniroma2.it>
% May 14, 2024

% --- Recap ---
\begin{frame}{Recap}
	\textbg{ROS 2} software is organized in \textbg{packages}, built by \textbg{\texttt{colcon}} invoking either \textbg{CMake} or \textbg{setuptools}.\\
	\bigskip
	\textbg{Messages} are the most basic, \textbg{one-way} communication paradigm.\\
	On the \textbg{DDS RMW}, ROS 2 topics directly resolve to \textbg{DDS topics}.\\
	\bigskip
	Messages formats are defined in \textbg{interface files} which usually constitute entire packages.\\
	\bigskip
	This lecture is \href{https://github.com/robmasocco/DAFN24_Robotics_3}{\color{blue}\underline{here}}.
\end{frame}
\begin{frame}{Recap}
	\begin{block}{Updates}
		\begin{itemize}
			\item Updated \textbf{lectures program}.
			\item Follow-up on \textbf{message topics code examples}:
			      \begin{itemize}
				      \item \href{https://github.com/IntelligentSystemsLabUTV/ros2-examples/tree/humble/src/cpp/topic_pubsub_cpp}{\color{blue}\underline{ros2-examples/src/cpp/topic\_pubsub\_cpp}}
				      \item CLI inspection tools.
			      \end{itemize}
		\end{itemize}
	\end{block}
\end{frame}
\begin{frame}{Recap}{Updated lectures program}
	\begin{enumerate}
		\item Roboticist 101 - Software and middleware for robotics
		\item ROS 2 - Workflow and basic communication
		\item \textbg{ROS 2 - Advanced communication I}
		\item \textbg{ROS 2 - Advanced communication II}
		\item ROS 2 - Node configuration
		\item ROS 2 - Sensor sampling and image processing
		\item Localization and mapping - From EKF to SLAM
		\item Inside the roboticist's toolbox - Linux kernel, Docker, and more
		\item microROS - Bridging the gap
		\item MARTe2 - A real-time control framework for nuclear fusion
	\end{enumerate}
\end{frame}
